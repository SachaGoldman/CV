\documentclass[]{style}

\begin{document}

\header{Sacha}{Goldman}{}% Subtitle: (Software Engineer, Machine Learning Engineer, Computer Science and Mathematics Student)

\begin{aside} % In the aside, each new line forces a line break
\section{Location}
Toronto, Canada
~ \vspace{-2mm}
US + Canadian Citizen 
Willing to Relocate
\section{Languages}
{\color{red} $\varheartsuit$} Python, Swift, 
C, TypeScript
\section{Tools}
PyTorch, Numpy, \\ \LaTeX, Git, Shell
\section{Online}
\textbf{Email}
\url{mailto:sachagoldman@icloud.com}{sachagoldman@icloud.com} 
~ \vspace{-2mm}
\textbf{Website} 
\url{https://sachagoldman.com}{sachagoldman.com} 
~ \vspace{-2mm}
\textbf{Github}
\url{https://github.com/SachaGoldman}{SachaGoldman}
~ \vspace{-2mm}
\textbf{LinkedIn}
\url{https://www.linkedin.com/in/sacha-goldman-5b95391a0/}{Sacha Goldman}
\section{Awards}
New College Council 
In-Course Scholarship
~ \vspace{-1mm}
William and Shirley Read 
Scholarship
~ \vspace{-1mm}
VSB District Scholarship
\end{aside}

\section{Education}

\begin{entrylist}

\entry
{Computer Science and Mathematics \ {\normalfont University of Toronto}}
{Bachelors of Science}
{\emph{3.83/4.0 GPA \ Graduating May 2023}
~ \vspace{1mm}

Focused on math and theoretical computer science, taking courses in probabilistic learning, deep learning, algorithms, data structures, probability theory, real analysis, calculus on manifolds, topology, linear algebra, and group theory.}

\end{entrylist}

\section{Research}

\begin{entrylist}

\vspace{2mm}

\entry
{Quantum Machine Learning \ {\normalfont University of Toronto}}
{Toronto, 2022}
{ ~ \vspace{-2.5mm}

Conducting research with Nathan Wiebe into bringing the core ideas of convolutional neural networks into the context of quantum machine learning. Specifically, trying to learn data with translation and scale invariant properties, and trying to avoid the vanishing gradient challenges presented by traditional quantum neural networks.
}

\end{entrylist}

\section{Experience}

\begin{entrylist}

\vspace{2mm}

\entry
{\url{https://www.apple.com}{Apple} \ {\normalfont Software Engineer Intern}}
{Remote/Cupertino, 2022}
{ ~ \vspace{-2.5mm}

\tile{Swift} \tile{Typescript} \tile{Frameworks}

Working on the team architecting Apple Media Apps. Developed my skills writing both Swift and Typescript to deliver value to many key services. 
}

\entry
{University of Toronto \ {\normalfont Teaching Assistant}}
{Toronto, 2021}
{ ~ \vspace{-2.5mm}

\tile{Tutorials} \tile{Marking} \tile{Theory of Computer Science}

Teaching assistant for CSC236, an introductory course to computer science theory. Taught two weekly tutorials, covering concepts like induction, automata, formal languages, and computational complexity. Also marked tests and assignments. 
}

%------------------------------------------------
%
%% The inclusion of this depends on the position
%
%\vspace{2mm}
%
%\entry
%{\url{https://www.ssense.com}{SSENSE}\vspace{1mm} \ {\normalfont Software Developer Intern}}
%{Remote, 2021}
%{ ~ \vspace{-2.5mm}
%
%\tile{Swift}\tile{Code Review} \tile{Unit Testing}
%
%Worked on the mobile team during my 4 month internship, brining a fresh set of ideas to the team, and advocating for a transition to the composable architecture along with the introduction of SwiftUI. Acted as a feature lead on a new image zoom experience. Spearheaded a rewrite of the main product page in SwiftUI, with greatly improved gestures.}

%------------------------------------------------
%
%% The inclusion of this depends on the position
%
%\vspace{2mm}
%
%\entry
%{Tutor\vspace{2mm}}
%{Toronto 2021}
%{Tutored UofT's CSC263, a course on data structures and algorithms. Helped a student understand difficult and nuanced concepts for the first time, by presenting them from a different perspective.}

\end{entrylist}

\section{Projects}

\begin{entrylist}

\vspace{2mm}

\entry
{K2}{macOS App}
{ ~ \vspace{-2.5mm}

\tile{Machine Learning} \tile{Python} \tile{Swift}

K2 improves upon Apple Photos' built-in facial clustering by scanning your photo library and creating an album of each unique face found. The application uses the Photos API to find the pictures, then runs python subprocesses which find the faces in each photo, using a SVM, and vectorize them, using a CNN. These feature vectors are then clustered using DBSCAN.
\vspace{1mm}

Hurdles overcome included tuning the finicky model hyperparameters, and code signing python for the Mac App Store.}

\vspace{2mm}

\entry
{\url{https://sachagoldman.com}{sachagoldman.com}}{Website}
{ ~ \vspace{-2.5mm}

\tile{Vue} \tile{TypeScript}

My personal website serving as a home page for my presence online. This website was created from scratch in Vue and showcases my projects and academics. 
\vspace{1mm}

Prevailed over the challenge of learning Vue, as it was a completely foreign the framework.}

\end{entrylist}

\end{document}