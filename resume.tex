\documentclass[]{style}

\begin{document}

\header{Sacha}{Goldman}{}% Subtitle: (Software Engineer, Machine Learning Engineer, Computer Science and Mathematics Student)

%----------------------------------------------------------------------------------------
%	SIDEBAR SECTION
%----------------------------------------------------------------------------------------

\begin{aside} % In the aside, each new line forces a line break
\section{Location}
Toronto, Canada
~ \vspace{-2mm}
US + Canadian Citizen 
Willing to Relocate
\section{Languages}
{\color{red} $\varheartsuit$} Python, Swift, 
C, TypeScript
\section{Tools}
PyTorch, Numpy, \\ \LaTeX, Git, Shell
\section{Online}
\textbf{Email}
\url{mailto:sachagoldman@icloud.com}{sachagoldman@icloud.com} 
~ \vspace{-2mm}
\textbf{Website} 
\url{https://sachagoldman.com}{sachagoldman.com} 
~ \vspace{-2mm}
\textbf{Github}
\url{https://github.com/SachaGoldman}{SachaGoldman}
~ \vspace{-2mm}
\textbf{LinkedIn}
\url{https://www.linkedin.com/in/sacha-goldman-5b95391a0/}{Sacha Goldman}
\section{Awards}
New College Council 
In-Course Scholarship
~ \vspace{-1mm}
William and Shirley Read 
Scholarship
~ \vspace{-1mm}
VSB District Scholarship
\end{aside}

%----------------------------------------------------------------------------------------
%	EDUCATION SECTION
%----------------------------------------------------------------------------------------

\section{Education}

\begin{entrylist}

%------------------------------------------------

\entry
{Computer Science and Mathematics \ {\normalfont University of Toronto}}
{Bachelors of Science}
{\emph{3.82/4.0 GPA \ Graduating May 2023}
~ \vspace{1mm}

Primarily interested in the theoretical aspects of machine learning, and how we can improve upon current paradigms. Taken courses in probabilistic learning and deep learning. Also fascinated by pure math. Taken courses on probability theory, real analysis, topology, differential geometry, linear algebra, and group theory. }

%------------------------------------------------

\end{entrylist}

%----------------------------------------------------------------------------------------
%	WORK EXPERIENCE SECTION
%----------------------------------------------------------------------------------------

\section{Experience}

\begin{entrylist}

%------------------------------------------------

\vspace{2mm}

\entry
{University of Toronto \ {\normalfont Teaching Assistant}}
{Toronto, 2021}
{ ~ \vspace{-2.5mm}

\tile{Tutorials} \tile{Marking} \tile{Theory of Computer Science}

Teaching assistant for CSC236, an introductory course to computer science theory. Taught two weekly tutorials, covering concepts like induction, automata, formal languages, and computational complexity. Also marked tests and assignments. 
}


\entry
{\url{https://www.ssense.com}{SSENSE}\vspace{1mm} \ {\normalfont Software Developer Intern}}
{Montreal, 2021}
{ ~ \vspace{-2.5mm}

\tile{Swift} \tile{SwiftUI} \tile{UIKit} \tile{Code Review} \tile{Unit Testing}

Worked on the mobile team during my 4 month internship, brining a fresh set of ideas to the team, and advocating for a transition to the composable architecture along with the introduction of SwiftUI. Acted as a feature lead on a new image zoom experience. Spearheaded a rewrite of the main product page in SwiftUI, with greatly improved gestures.}

%------------------------------------------------
%
%% The inclusion of this depends on the position
%
%\vspace{2mm}
%
%\entry
%{Tutor\vspace{2mm}}
%{Toronto 2021}
%{Tutored UofT's CSC263, a course on data structures and algorithms. Helped a student understand difficult and nuanced concepts for the first time, by presenting them from a different perspective.}

%------------------------------------------------

\entry
{\url{https://www.altairix.com/}{Altairix} \vspace{1mm} \ {\normalfont Software Developer Intern}}
{Toronto, 2020}
{ ~ \vspace{-2.5mm}

\tile{Java} \tile{SQL} \tile{Statistical Analysis}

Developed the web and Android based learning platforms for the Arrowsmith Program during my 4 month internship. Created new ways to present student data to teachers through innovative student reports and interactive graphs.}

\end{entrylist}

%----------------------------------------------------------------------------------------
%	PROJECTS SECTION
%----------------------------------------------------------------------------------------

\section{Projects}

\begin{entrylist}

%------------------------------------------------

\vspace{2mm}

\entry
{\url{https://apps.apple.com/us/app/k2/id1572547510}{K2}}{macOS App}
{ ~ \vspace{-2.5mm}

\tile{Machine Learning} \tile{Python} \tile{Swift} \tile{SwiftUI}

K2 improves upon Apple Photos' built-in facial clustering by scanning your photo library and creating an album of each unique face found. The application uses the Photos API to find the pictures, then runs python subprocesses which find the faces in each photo, using a SVM, and vectorize them, using a CNN. These vectors are then clustered using DBSCAN.
\vspace{1mm}

Hurdles overcome included tuning the finicky model hyperparameters, and code signing python for the Mac App Store.}

%------------------------------------------------

\vspace{2mm}

\entry
{\url{https://sachagoldman.com}{sachagoldman.com}}{Website}
{ ~ \vspace{-2.5mm}

\tile{Vue} \tile{TypeScript}

My personal website serving as a home page for my presence online. This website was created from scratch in Vue and showcases my projects and academics. 
\vspace{1mm}

Prevailed over the challenge of learning Vue, as it was a completely foreign the framework.}

%------------------------------------------------

% The inclusion of this depends on the position

%\entry
%{\url{https://sachagoldman.com/dots/}{Dots}}{Web App}
%{ ~ \vspace{-2.5mm}
%
%\tile{JavaScript}
%
%Developed a simulation of a number of particles, or dots, that apply forces on each other to form interesting, natural looking structures. Fun and entertaining.}

%------------------------------------------------

\end{entrylist}

\end{document}