\documentclass[]{style}

\begin{document}

\header{Sacha}{Goldman}{}% Subtitle: (Software Engineer, Machine Learning Engineer, Computer Science and Mathematics Student)

%----------------------------------------------------------------------------------------
%	SIDEBAR SECTION
%----------------------------------------------------------------------------------------

\begin{aside} % In the aside, each new line forces a line break
\section{Location}
Toronto, Canada
~ \vspace{-2mm}
US + Canadian Citizen 
Willing to Relocate
\section{Languages}
{\color{red} $\varheartsuit$} Python, Swift, 
C, TypeScript
\section{Tools}
\LaTeX, Git, Shell, 
SwiftUI, PyTorch
\section{Online}
\textbf{Email}
\url{mailto:sachagoldman@icloud.com}{sachagoldman@icloud.com} 
~ \vspace{-2mm}
\textbf{Website} 
\url{https://sachagoldman.com}{sachagoldman.com} 
~ \vspace{-2mm}
\textbf{Github}
\url{https://github.com/SachaGoldman}{SachaGoldman}
~ \vspace{-2mm}
\textbf{LinkedIn}
\url{https://www.linkedin.com/in/sacha-goldman-5b95391a0/}{Sacha Goldman}
\section{Awards}
New College Council 
In-Course Scholarship
~ \vspace{-1mm}
William and Shirley Read 
Scholarship
~ \vspace{-1mm}
VSB District Scholarship
\end{aside}

%----------------------------------------------------------------------------------------
%	EDUCATION SECTION
%----------------------------------------------------------------------------------------

\section{Education}

\begin{entrylist}

%------------------------------------------------

\entry
{Computer Science and Mathematics \ {\normalfont University of Toronto}}
{Bachelors of Science}
{\emph{3.85/4.0 GPA \ Graduating May 2023}
~ \vspace{1mm}

Primarily interested in the applying both mathematics and computer science to machine learning. Core topics include: algorithms, data-structures, linear algebra, group theory, single and multivariable analysis, probability and statistics, and programming.}

%------------------------------------------------

\end{entrylist}

%----------------------------------------------------------------------------------------
%	WORK EXPERIENCE SECTION
%----------------------------------------------------------------------------------------

\section{Experience}

\begin{entrylist}

%------------------------------------------------

\vspace{2mm}

\entry
{\url{https://www.ssense.com}{SSENSE}\vspace{1mm}}
{Montreal, 2021}
{ ~ \vspace{-2.5mm}

\tag{Swift} \tag{SwiftUI} \tag{UIKit} \tag{Code Review} \tag{Tests}

Worked as an iOS developer on the mobile team during my 4 month internship. Brought a fresh set of an ideas to the team, and advocated for 
a transition to the composable architecture and introduction of SwiftUI. Acted as a feature lead to research and create new features, including a product image zoom experience and better ApplePay integration.}

%------------------------------------------------
%
%% The inclusion of this depends on the position
%
%\vspace{2mm}
%
%\entry
%{Data Structures Tutor\vspace{2mm}}
%{Toronto 2021}
%{Tutored UofT's CSC263, a course on data structures and algorithms. Helped a student understand difficult and nuanced concepts for the first time, by presenting them from a different perspective.}

%------------------------------------------------

\entry
{\url{https://www.altairix.com/}{Altairix} \vspace{1mm}}
{Toronto, 2020}
{ ~ \vspace{-2.5mm}

\tag{Java} \tag{SQL} \tag{Statistical Analysis}

Worked as a full stack developer, creating the web and Android based learning platforms for the Arrowsmith Program during my 4 month internship. Created new ways to present student data to teachers, such as, innovative student reports and interactive graphs.}

\end{entrylist}


%----------------------------------------------------------------------------------------
%	PROJECTS SECTION
%----------------------------------------------------------------------------------------

\section{Projects}

\begin{entrylist}

%------------------------------------------------

\vspace{2mm}

\entry
{\url{https://apps.apple.com/us/app/k2/id1572547510}{K2}}{macOS App}
{ ~ \vspace{-2.5mm}

\tag{Machine Learning} \tag{Python} \tag{Swift} \tag{SwiftUI}

K2 improves upon Apple Photos' built-in facial clustering by scanning your photo library and creating an album of each unique face found. The application uses the Photos API to find the pictures, then runs up python subprocesses which finds faces in each photo using a SVM and vectorizes them using a CNN. These vectors are then clustered using DBSCAN.
\vspace{1mm}

Hurdles overcome included code signing python for the Mac App Store, and tuning the model to get the best accuracy.}

%------------------------------------------------

\vspace{2mm}

\entry
{\url{https://sachagoldman.com}{sachagoldman.com}}{Website}
{ ~ \vspace{-2.5mm}

\tag{Vue} \tag{TypeScript}

My personal website serving as a home page for my presence online. This website was created from scratch in Vue and showcases my projects and academics. 
\vspace{1mm}

Prevailed over the challenge of learning Vue, as it was a completely foreign the framework.}

%------------------------------------------------

% The inclusion of this depends on the position

\entry
{\url{https://sachagoldman.com/dots/}{Dots}}{Web App}
{ ~ \vspace{-2.5mm}

\tag{JavaScript}

Developed a simulation of a number of particles, or dots, that apply forces on each other to form interesting, natural looking structures. Fun and entertaining.}

%------------------------------------------------

\end{entrylist}

\end{document}