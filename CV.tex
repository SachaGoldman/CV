\documentclass[]{style}

\begin{document}

\header{Sacha}{Goldman}{}% Subtitle: (Software Engineer, Machine Learning Engineer, Computer Science and Mathematics Student)

\begin{aside} % In the aside, each new line forces a line break
\section{Location}
Toronto, Canada
\section{Technical Tools}
Lean, Python, C, PyTorch, \\ Numpy, \LaTeX, Git, Shell
\section{Online}
\textbf{Email}
\url{mailto:sacha@sachagoldman.com}{sacha@sachagoldman.com} 
~ \vspace{-2mm}
\textbf{Website} 
\url{https://sachagoldman.com}{sachagoldman.com} 
~ \vspace{-2mm}
\textbf{Github}
\url{https://github.com/SachaGoldman}{SachaGoldman}
\section{Awards}
New College Council 
In-Course Scholarship
~ \vspace{-1mm}
William and Shirley Read 
Scholarship
~ \vspace{-1mm}
VSB District Scholarship
\end{aside}

\section{Education}x

\begin{entrylist}

\entry
{Computer Science and Mathematics \ {\normalfont University of Toronto}}
{Bachelors of Science}
{\emph{3.85/4.0 GPA \ Graduating May 2023}
~ \vspace{1mm}

Focused on math and theoretical computer science, studying differential topology, complex analysis, riemannian geometry, probabilistic learning, and quantum computing, and more.}

\end{entrylist}

\section{Research}

\begin{entrylist}

\vspace{2mm}

\entry
{Quantum Machine Learning \ {\normalfont University of Toronto}}
{Toronto, 2022}
{ ~ \vspace{-3.5mm}

Conducting research with Nathan Wiebe into bringing the core mathematical ideas of convolutional neural networks into the context of quantum machine learning. Specifically, trying to learn data with translation and scale invariant properties, and trying to avoid the vanishing gradient challenges presented by traditional quantum neural networks.
}

\end{entrylist}

\section{Employment}

\begin{entrylist}

\vspace{2mm}

\entry
{\url{https://www.apple.com}{Apple} \ {\normalfont Software Engineer Intern}}
{Remote/Cupertino, 2022}
{ ~ \vspace{-2.5mm}

\tile{Swift} \tile{Typescript} \tile{Frameworks}

Working on the team architecting Apple Media Apps. Collaborating with groups of extremely talented and diverse people to solve difficult technical challenges. 
}

\vspace{2mm}


\entry
{University of Toronto \ {\normalfont Teaching Assistant}}
{Toronto, 2021}
{ ~ \vspace{-2.5mm}

\tile{Tutorials} \tile{Marking} \tile{Proofs}

Teaching assistant for CSC236, an introductory course to computer science theory. Taught two weekly tutorials, covering concepts like induction, automata, formal languages, and computational complexity. Also marked tests and assignments. 
}

\vspace{2mm}

\entry
{\url{https://www.ssense.com}{SSENSE} \ {\normalfont Software Developer Intern}}
{Remote, 2021}
{ ~ \vspace{-2.5mm}

\tile{Swift} \tile{Code Review}

Worked on the iOS team during my 4 month internship, brining fresh ideas to the team and advocating for a transition to the composable architecture and SwiftUI. Acted as a feature lead on new features including a rewrite of the main product page in SwiftUI.}


\entry
{Tutor}
{Toronto 2021}
{ ~ \vspace{-2.5mm}

\tile{Proofs} \tile{Data Structures}

Tutored UofT's CSC263, a course on data structures and algorithms. Helped a student understand difficult and nuanced concepts by presenting them from a different perspective.}

\end{entrylist}

\section{Projects}

\begin{entrylist}

\vspace{2mm}

\entry
{K2}{macOS App}
{ ~ \vspace{-2.5mm}

\tile{Machine Learning} \tile{Python}

K2 improves upon Apple Photos' built-in facial clustering by scanning your photo library and creating an album of each unique face found. The application uses the Photos API to find the pictures, then runs python subprocesses which finds the faces in each photo using a SVM, and vectorizes them using a CNN. These feature vectors are then clustered using DBSCAN.}

\end{entrylist}
\end{document}